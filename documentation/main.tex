\documentclass{assignment}
\usepackage{arch_template} % Required for inserting images

\student{Natalie Chmura}
\semester{2024}
\date{\today}

\usepackage{amssymb}
\usepackage{tikz}
\tikzset{
  LabelStyle/.style = { rectangle, rounded corners, draw,
                        minimum width = 2em, fill = yellow!50,
                        text = red, font = \bfseries },
  VertexStyle/.append style = { inner sep=5pt,
                                font = \normalsize\bfseries},
  EdgeStyle/.append style = {->, bend left} }
\usetikzlibrary {positioning}
\usetikzlibrary{shapes}
\usetikzlibrary{automata}
\tikzset{
mystyle/.style={
  circle,
  inner sep=0pt,
  text width=6mm,
  align=center,
  draw=black,
  fill=white
  },
invis/.style={
    circle,
    inner sep=0pt,
    text width=6mm,
    align=center,
    draw=white,
    fill=white
}
}


\courselabel{Individual Studies}
\exercisesheet{Research}{Minimum Spanning Tree Rotation Transforms}

\school{Arts and Sciences}
\university{\textbf{THE} Ohio State University}

\begin{document}

\begin{problem}

\section*{Preface}

\noindent The premise of this paper is to derive an algorithm  that finds the minimum number of right and left rotations that transforms one Binary Search Tree into a transformed version (Same nodes, different structure)

\subsection*{Background and Assumptions} \\

\noindent For the sake of simplification, any values in the binary search tree will be values in $\mathbb{Z}$, and will use standard ordering. While there may be an additional proof \cite{HERE} about finding the minimum number of rotations given duplicate elements in the trees, for this problem we will consider duplicates to be invalid. \\

\noindent The right and left rotations will be performed as followed: \\


\begin{center}

Right Rotate on \{x\}

\begin{tikzpicture}
\small
    \node[mystyle] (1) at (-3, 0) {p};
    \node[mystyle] (2) at (3, 0)  {p};
    \node[invis] (3) at (-2, 0) {};
    \node[invis] (4) at (2, 0) {};
    \node[invis] (5) at (-3, 1) {};
    \node[invis] (6) at (3, 1) {};
    
    \node[mystyle] (7) at (-3, -1.25) {x};
    \node[mystyle] (8) at (3, -1.25) {y};
    
    \node[mystyle] (9) at (-3.75, -2.50) {y};
    \node[invis] (10) at (-2.25, -2.50) {c};
    
    \node[invis] (11) at (-4.5, -3.75) {a};
    \node[invis] (12) at (-3, -3.75) {b};
    
    \node[mystyle] (13) at (3.75, -2.50) {x};
    \node[invis] (14) at (2.25, -2.50) {a};

    \node[invis] (15) at (4.5, -3.75) {c};
    \node[invis] (16) at (3, -3.75) {b};

    \node[invis] (17) at (-2, -4) {};
    \node[invis] (18) at (2, -4) {};
    
    
    \draw[->] (3) edge[out=30,in=150,looseness=1] (4);
    \draw[->] (18) edge[out=210, in=330, looseness=1] (17);
    
    \draw (1) edge[out = 90, in=270] (5);
    \draw (2) edge[out = 90, in=270] (6);
    \draw (1) edge[out = 270, in=90] (7);
    \draw (2) edge[out = 270, in=90] (8);

    \draw (7) edge[out=240, in=60] (9);
    \draw (7) edge[out=300, in=120] (10);

    \draw (9) edge[out=240, in=60] (11);
    \draw (9) edge[out=300, in=120] (12);

    \draw (8) edge[out=240, in=60] (14);
    \draw (8) edge[out=300, in=120] (13);

    \draw (13) edge[out=240, in=60] (16);
    \draw (13) edge[out=300, in=120] (15);

    
\end{tikzpicture}

Left Rotate on \{y\}
\end{center}


\end{problem}

\end{document}
